\documentclass[11pt,a4paper]{article}

\usepackage[margin=25mm]{geometry}
\usepackage{fancyhdr}
\usepackage{enumitem}
\usepackage[colorlinks=true,linkcolor=blue!60!black,urlcolor=blue!60!black]{hyperref}
\usepackage{xcolor}
\usepackage{longtable}
\usepackage{booktabs}
\usepackage{listings}
\usepackage{parskip}
\usepackage{lastpage}
\usepackage{graphicx}
\usepackage{textcomp}

% ── Page style ──
\pagestyle{fancy}
\fancyhf{}
\renewcommand{\headrulewidth}{0.4pt}
\fancyhead[L]{\small Hydrogen Line Beacon}
\fancyhead[R]{\small Mikoshi Ltd}
\fancyfoot[C]{\small Hydrogen Line Beacon --- Build Guide --- Mikoshi Ltd --- Page \thepage\ of \pageref{LastPage}}
\renewcommand{\footrulewidth}{0.4pt}

% ── Listings ──
\lstset{
  basicstyle=\ttfamily\small,
  breaklines=true,
  breakatwhitespace=false,
  frame=single,
  backgroundcolor=\color{gray!8},
  rulecolor=\color{gray!40},
  xleftmargin=4pt,
  xrightmargin=4pt,
  aboveskip=8pt,
  belowskip=8pt,
  columns=fullflexible,
  keepspaces=true,
  literate={→}{{$\rightarrow$}}1
           {←}{{$\leftarrow$}}1
           {↑}{{$\uparrow$}}1
           {✓}{{$\checkmark$}}1
           {✗}{{x}}1
           {▼}{{v}}1
           {●}{{*}}1
}

\lstdefinestyle{diagram}{
  basicstyle=\ttfamily\footnotesize,
  frame=single,
  backgroundcolor=\color{gray!5},
  xleftmargin=4pt,
  xrightmargin=4pt,
}

\begin{document}

% ── Title ──
\begin{center}
{\LARGE\bfseries Hydrogen Line Beacon}\\[6pt]
{\Large Complete Build Guide}\\[12pt]
{\large Full hardware assembly, wiring, and deployment instructions.}\\[12pt]
{\large\textbf{Mikoshi Ltd --- February 2026}}\\[6pt]
\end{center}

\tableofcontents
\newpage

%% ============================================================
\section{Overview}

\subsection{What We're Building}

A portable, field-deployable dual-channel signal system controlled by a Raspberry Pi~5:

\begin{lstlisting}[style=diagram]
+-------------------------------------------------------------+
|                    RASPBERRY PI 5                            |
|                                                              |
|  +----------+  +----------+  +----------+  +-----------+    |
|  | Protocol |  | Waveform |  | Monitor  |  |  Logger   |    |
|  | Engine   |  | Generator|  | (SDR RX) |  |  (JSON)   |    |
|  +----+-----+  +----+-----+  +----+-----+  +-----------+    |
|       |              |              |                         |
+-------+--------------+--------------+------------------------+
        |              |              |
   +----v----+    +----v----+   +----v----+
   | USB 3.0 |    |  I2S    |   | USB 2.0 |
   +----+----+    +----+----+   +----+----+
        |              |              |
   +----v----+    +----v----+   +----v----+
   | HackRF  |    |PCM5102A |   | RTL-SDR |
   |  One    |    |  DAC    |   |   v4    |
   +----+----+    +----+----+   +----+----+
        |              |              |
   +----v----+    +----v----+   +----v----+
   |Helical  |    |TPA3116  |   |Discone  |
   |Antenna  |    |  Amp    |   |Antenna  |
   |(TX)     |    +----+----+   |(RX)     |
   +---------+    +----v----+   +---------+
                  |  Bass   |
        EM        | Shaker  |      EM
      Radiation   +----+----+   Reception
                  +----v----+
                  | Ground  |
                  |  Plate  |
                  +----+----+
                  +----v----+
                  | Ground  |
                  |  Spike  |
                  +---------+
                   Seismic
                    Wave
\end{lstlisting}

\subsection{What Goes Where}

\begin{longtable}{llll}
\toprule
\textbf{Component} & \textbf{Connects To} & \textbf{Channel} & \textbf{Purpose} \\
\midrule
\endhead
Raspberry Pi 5        & Everything       & Controller  & Runs all software \\
HackRF One            & Pi USB 3.0       & RF TX       & Transmits hydrogen line signal \\
Helical antenna       & HackRF SMA       & RF TX       & Radiates EM signal \\
PCM5102A DAC          & Pi GPIO (I2S)    & Mechanical  & Converts digital to analogue \\
TPA3116D2 amp         & DAC output       & Mechanical  & Amplifies signal for shaker \\
Bass shaker           & Amp output       & Mechanical  & Converts electrical to vibration \\
Ground plate          & Shaker mount     & Mechanical  & Transfers vibration to earth \\
Ground spike          & Plate underside  & Mechanical  & Couples to soil \\
RTL-SDR v4            & Pi USB 2.0       & Monitor     & Receives EM for analysis \\
Discone antenna       & RTL-SDR SMA      & Monitor     & Wideband EM reception \\
USB magnetometer      & Pi USB           & Monitor     & Magnetic field anomalies \\
Pi Camera             & Pi CSI           & Monitor     & Sky observation \\
\bottomrule
\end{longtable}


%% ============================================================
\section{Bill of Materials}

\subsection{Tier 1: Mechanical Only (\textasciitilde\pounds150)}

No RF, no licence needed. Ground transduction only.

\begin{longtable}{clllr}
\toprule
\textbf{\#} & \textbf{Item} & \textbf{Model / Spec} & \textbf{Where to Buy} & \textbf{Est.\ Cost} \\
\midrule
\endhead
1  & Raspberry Pi 5 (4GB)  & BCM2712, 4GB RAM           & The Pi Hut / Pimoroni     & \pounds50 \\
2  & Pi 5 power supply     & USB-C 27W (5V/5A)          & The Pi Hut                & \pounds12 \\
3  & MicroSD card          & 32GB+ Class 10 / A2         & Amazon                    & \pounds8 \\
4  & I2S DAC board         & PCM5102A breakout            & Amazon / AliExpress       & \pounds8 \\
5  & Class D amplifier     & TPA3116D2 2-ch 50W           & Amazon                    & \pounds12 \\
6  & 12V power supply      & 12V 3A DC adapter            & Amazon                    & \pounds8 \\
7  & Bass shaker           & Dayton Audio BST-1 (10W)     & Parts Express / Amazon    & \pounds25 \\
8  & Ground plate          & 400mm aluminium disc, 6mm    & eBay / metals4U           & \pounds20 \\
9  & Ground spike          & 300mm steel rod, 12mm dia    & Screwfix                  & \pounds5 \\
10 & Speaker wire          & 14 AWG, 5m                   & Amazon                    & \pounds6 \\
11 & Jumper wires          & F--F dupont, 20cm            & Amazon                    & \pounds3 \\
12 & M4 bolts + nuts       & For mounting shaker          & Screwfix                  & \pounds3 \\
   &                       &                              & \textbf{Total}            & \textbf{\textasciitilde\pounds160} \\
\bottomrule
\end{longtable}

\subsection{Tier 2: Full Build (\textasciitilde\pounds530)}

Everything above, plus RF and monitoring.

\begin{longtable}{clllr}
\toprule
\textbf{\#} & \textbf{Item} & \textbf{Model / Spec} & \textbf{Where to Buy} & \textbf{Est.\ Cost} \\
\midrule
\endhead
13 & HackRF One            & Great Scott Gadgets, 1--6\,GHz & Lab401 / Amazon       & \pounds250 \\
14 & Helical antenna       & 1.42\,GHz RHCP, SMA            & eBay / DIY            & \pounds40 \\
15 & SMA cable (TX)        & SMA M--M, 1m, RG316             & Amazon                & \pounds8 \\
16 & RTL-SDR v4            & RTL2832U + R828D, SMA            & rtl-sdr.com           & \pounds30 \\
17 & Discone antenna       & 25\,MHz--1.3\,GHz wideband       & Amazon / Moonraker    & \pounds40 \\
18 & SMA cable (RX)        & SMA M--M, 1m, RG316             & Amazon                & \pounds8 \\
19 & USB magnetometer      & RM3100 or HMC5883L              & Amazon / AliExpress   & \pounds15 \\
20 & Pi Camera Module 3    & 12MP, autofocus, wide            & The Pi Hut            & \pounds30 \\
   &                       &                                  & \textbf{Tier 2 adds}  & \textbf{\textasciitilde\pounds421} \\
   &                       &                                  & \textbf{Full total}   & \textbf{\textasciitilde\pounds530} \\
\bottomrule
\end{longtable}

\subsection{Tier 3: Field Portable (\textasciitilde\pounds600)}

Everything above, plus battery power and case.

\begin{longtable}{clllr}
\toprule
\textbf{\#} & \textbf{Item} & \textbf{Model / Spec} & \textbf{Where to Buy} & \textbf{Est.\ Cost} \\
\midrule
\endhead
21 & 12V LiFePO4 battery   & 6Ah+ (powers amp + Pi)     & Amazon                & \pounds40 \\
22 & 12V to 5V USB-C conv. & Buck converter for Pi      & Amazon                & \pounds8 \\
23 & Pelican-style case     & IP67, \textasciitilde400$\times$300$\times$150mm & Amazon / Screwfix & \pounds30 \\
24 & Cable glands           & PG9/PG11, waterproof       & Amazon                & \pounds5 \\
   &                        &                            & \textbf{Field adds}   & \textbf{\textasciitilde\pounds83} \\
   &                        &                            & \textbf{Grand total}  & \textbf{\textasciitilde\pounds613} \\
\bottomrule
\end{longtable}


%% ============================================================
\section{Tier 1: Mechanical Only Build}

This is the simplest build. No RF, no licence, no SDR.

\subsection{What You Get}

\begin{itemize}
  \item Schumann resonance frequencies (7.83--33.8\,Hz) coupled directly into the ground
  \item Prime-number pulse timing
  \item Multiple signal programmes
  \item Full protocol logging
\end{itemize}

\subsection{Signal Path}

\begin{lstlisting}
Pi 5 GPIO (I2S) -> PCM5102A DAC -> TPA3116D2 Amp -> Bass Shaker -> Ground Plate -> Earth
\end{lstlisting}


%% ============================================================
\section{Tier 2: Full Build}

Adds RF transmission, EM monitoring, and anomaly detection.

\subsection{What You Get}

Everything in Tier~1, plus:

\begin{itemize}
  \item 1.42\,GHz hydrogen line transmission (or ISM-legal 433/868\,MHz)
  \item Schumann AM modulation on RF carrier
  \item Wideband EM monitoring with anomaly detection
  \item Magnetic field monitoring
  \item Sky camera
  \item Call-and-response protocol with automatic adaptation
\end{itemize}


%% ============================================================
\section{Raspberry Pi 5 Setup}

\subsection{Flash the OS}

\begin{enumerate}
  \item Download \textbf{Raspberry Pi Imager} from raspberrypi.com
  \item Insert microSD card into your computer
  \item Select \textbf{Raspberry Pi OS (64-bit, Bookworm)}
  \item Click the gear icon and configure:
    \begin{itemize}
      \item Hostname: \texttt{hlb}
      \item Enable SSH (password or key)
      \item Set username: \texttt{pi}
      \item Set password
      \item Configure WiFi (your network SSID + password)
    \end{itemize}
  \item Write to SD card
  \item Insert SD into Pi~5, connect power
\end{enumerate}

\subsection{First Boot}

\begin{lstlisting}[language=bash]
# SSH in
ssh pi@hlb.local
# or
ssh pi@<ip-address>

# Update system
sudo apt update && sudo apt upgrade -y

# Install system dependencies
sudo apt install -y \
    python3-pip \
    python3-numpy \
    python3-scipy \
    git \
    alsa-utils \
    libportaudio2

# Enable I2S (for DAC)
sudo raspi-config
# -> Interface Options -> I2S -> Enable
# Reboot when prompted
\end{lstlisting}

\subsection{Enable I2S Audio Output}

Edit the boot config:

\begin{lstlisting}[language=bash]
sudo nano /boot/firmware/config.txt
\end{lstlisting}

Add these lines at the end:

\begin{lstlisting}
# I2S DAC output
dtoverlay=hifiberry-dac
dtoverlay=i2s-mmap
\end{lstlisting}

Edit ALSA config to use the DAC:

\begin{lstlisting}[language=bash]
sudo nano /etc/asound.conf
\end{lstlisting}

Add:

\begin{lstlisting}
pcm.!default {
    type hw
    card 0
}

ctl.!default {
    type hw
    card 0
}
\end{lstlisting}

Reboot:

\begin{lstlisting}[language=bash]
sudo reboot
\end{lstlisting}

Verify the DAC appears:

\begin{lstlisting}[language=bash]
aplay -l
# Should show the I2S/HiFiBerry device
\end{lstlisting}


%% ============================================================
\section{Software Installation}

\subsection{Clone the Repository}

\begin{lstlisting}[language=bash]
cd ~
git clone https://github.com/DarrenEdwards111/UAP_Dog_Whistle.git
cd UAP_Dog_Whistle/hydrogen-line-beacon
\end{lstlisting}

\subsection{Install the Package}

\begin{lstlisting}[language=bash]
pip3 install -e . --break-system-packages
\end{lstlisting}

\subsection{Verify Installation}

\begin{lstlisting}[language=bash]
hlb --check
\end{lstlisting}

Expected output (Tier~1):
\begin{lstlisting}
Hardware check:
  HackRF One:    x Not found
  RTL-SDR:       x Not found
  Audio (aplay): OK Available
\end{lstlisting}

\subsection{Install HackRF Tools (Tier 2 only)}

\begin{lstlisting}[language=bash]
sudo apt install -y hackrf libhackrf-dev

# Verify
hackrf_info
# Should show "Found HackRF" with serial number
\end{lstlisting}

\subsection{Install RTL-SDR Tools (Tier 2 only)}

\begin{lstlisting}[language=bash]
sudo apt install -y rtl-sdr librtlsdr-dev

# Blacklist default DVB-T driver (conflicts with SDR use)
echo "blacklist dvb_usb_rtl28xxu" | sudo tee /etc/modprobe.d/blacklist-rtl.conf
sudo modprobe -r dvb_usb_rtl28xxu

# Verify
rtl_test -t
# Should show "Found 1 device(s)"
\end{lstlisting}


%% ============================================================
\section{Channel 1: Mechanical Assembly}

\subsection{Wiring the DAC}

The PCM5102A connects to the Pi~5's GPIO header via I2S:

\begin{lstlisting}[style=diagram]
PCM5102A Board          Raspberry Pi 5 GPIO
--------------          -------------------
VIN  ------------------- Pin 1  (3.3V)
GND  ------------------- Pin 6  (GND)
BCK  ------------------- Pin 12 (GPIO 18 -- I2S BCLK)
LCK  ------------------- Pin 35 (GPIO 19 -- I2S LRCLK)
DIN  ------------------- Pin 40 (GPIO 21 -- I2S DOUT)
\end{lstlisting}

Some PCM5102A boards also have:
\begin{lstlisting}[style=diagram]
FLT  ------------------- GND (normal filter)
DMP  ------------------- GND (de-emphasis off)
FMT  ------------------- GND (I2S format)
SCL  ------------------- Leave unconnected (or GND)
XMT  ------------------- 3.3V (unmute)
\end{lstlisting}

\subsection{Wiring the Amplifier}

\begin{lstlisting}[style=diagram]
PCM5102A                TPA3116D2 Amplifier
--------                -------------------
L OUT ------------------- L IN
GND   ------------------- GND

12V DC Supply           TPA3116D2 Amplifier
-------------           -------------------
+12V  ------------------- VCC / V+
GND   ------------------- GND

TPA3116D2 Amplifier     Bass Shaker
-------------------     -----------
L OUT+ ------------------- + (red)
L OUT- ------------------- - (black)
\end{lstlisting}

\textbf{Important:} The amplifier runs on 12V DC, NOT from the Pi's 5V. Use a separate 12V power supply or 12V battery.

\subsection{Assembling the Ground Coupler}

\begin{lstlisting}[style=diagram]
        +---------------------+
        |    Bass Shaker      |
        |   (Dayton BST-1)    |
        |                     |
        |  +---+   +---+     |  <- 4x M4 bolts through
        |  |M4 |   |M4 |     |     shaker mounting holes
        |  +---+   +---+     |
        +---------------------+
                |
        +-------v-------------+
        |                     |
        |   Aluminium Plate   |  <- 400mm diameter, 6mm thick
        |   (ground plate)    |     Drill 4x M4 holes to match
        |                     |
        +----------+----------+
                   |
              +----v----+
              | Ground  |  <- 300mm steel rod, 12mm dia
              |  Spike  |     Welded/bolted to plate centre
              |         |     Drives into soil
              +---------+
\end{lstlisting}

\textbf{Assembly steps:}

\begin{enumerate}
  \item \textbf{Drill mounting holes} in the aluminium plate to match the bass shaker's bolt pattern (usually 4 holes in a square).
  \item \textbf{Bolt the shaker} to the centre of the plate using M4 bolts, washers, and lock nuts. Tighten firmly---vibration will loosen anything that isn't locked.
  \item \textbf{Attach the ground spike} to the underside of the plate:
    \begin{itemize}
      \item \textbf{Option A (best):} Weld the steel rod perpendicular to the plate centre
      \item \textbf{Option B:} Drill a 12mm hole through the plate, push the rod through, secure with nuts on both sides
      \item \textbf{Option C:} Use a heavy-duty U-bolt to clamp the rod to the plate underside
    \end{itemize}
  \item \textbf{Connect speaker wire} from the amplifier to the shaker terminals.
\end{enumerate}

\subsection{Testing the Mechanical Channel}

\begin{lstlisting}[language=bash]
# Generate a 10-second test file
hlb --generate /tmp/test.wav --duration 10 --programme fundamental

# Play it
aplay /tmp/test.wav

# You should feel 7.83 Hz vibration through the plate
# It won't be audible -- it's below human hearing
# Place your hand on the plate to feel it
\end{lstlisting}


%% ============================================================
\section{Channel 2: RF Assembly}

\subsection{HackRF One Connection}

Simple---USB cable:

\begin{lstlisting}[style=diagram]
Raspberry Pi 5          HackRF One
--------------          ----------
USB 3.0 port ---------- USB-C port (use the blue USB port on Pi)
\end{lstlisting}

\subsection{Helical Antenna for 1.42\,GHz}

A circularly polarised helical antenna is ideal for hydrogen line work. You can buy one or build one.

\textbf{DIY Helical Antenna (1.42\,GHz, RHCP):}

\begin{lstlisting}[style=diagram]
Materials:
  - 2mm copper wire, ~2m length
  - PVC pipe, 67mm diameter, 300mm long
  - SMA female chassis connector
  - Ground plane: 150mm x 150mm aluminium sheet
  - 4mm wooden dowel (winding guide)

Dimensions (for 1.42 GHz):
  Circumference:    C = lambda = 211mm (wire on 67mm PVC pipe)
  Spacing:          S = lambda/4 = 53mm between turns
  Number of turns:  N = 6 (gives ~12 dBi gain)
  Total height:     H = N x S = 318mm
  Wire length:      L = N x C = 1,266mm
  Ground plane:     >= 0.75 lambda x 0.75 lambda = 158mm x 158mm
\end{lstlisting}

\begin{lstlisting}[style=diagram]
        ^ Direction of radiation
        |
    +---+---+
    | +-+-+ |   <- Turn 6 (top)
    | | | | |
    | +-+-+ |   <- Turn 5
    | | | | |
    | +-+-+ |   <- Turn 4
    | | | | |        53mm spacing
    | +-+-+ |   <- Turn 3
    | | | | |
    | +-+-+ |   <- Turn 2
    | | | | |
    | +-+-+ |   <- Turn 1 (feed)
    |   |   |
    +---+---+
   +----+--------+
   |    *        |  <- SMA connector (centre pin to wire)
   |  Ground     |  <- 150mm x 150mm aluminium
   |  Plane      |
   +-------------+
\end{lstlisting}

\textbf{Or buy:} Search eBay/Amazon for ``1420\,MHz helical antenna'' or ``hydrogen line antenna''---radio astronomy hobbyists sell these for \pounds30--60.

\subsection{SMA Connections}

\begin{lstlisting}[style=diagram]
HackRF One (SMA female) ---- SMA cable (M-M, 1m) ---- Antenna (SMA female)
\end{lstlisting}

Use RG316 or RG58 cable. Keep it short---every metre of cable loses \textasciitilde0.5\,dB at 1.4\,GHz.

\subsection{Testing the RF Channel}

\begin{lstlisting}[language=bash]
# Generate a 10-second baseband file
hlb --generate-rf /tmp/test.iq

# Verify HackRF
hackrf_info

# Test transmit (433 MHz ISM -- legal, safe)
hackrf_transfer -t /tmp/test.iq -f 433920000 -s 2000000 -x 10

# Stop with Ctrl+C
\end{lstlisting}

\textbf{Warning:} Do not transmit on 1.42\,GHz without a ham licence. Use 433\,MHz for testing.


%% ============================================================
\section{Monitoring Station Assembly}

\subsection{RTL-SDR Receiver}

\begin{lstlisting}[style=diagram]
Raspberry Pi 5          RTL-SDR v4
--------------          ----------
USB 2.0 port ---------- USB-A connector

RTL-SDR v4 (SMA)        Discone Antenna
-----------------        ---------------
SMA female ------------- SMA cable ----- Antenna base
\end{lstlisting}

The discone antenna should be mounted as high as possible---ideally on a tripod or mast, minimum 2m above ground.

\subsection{USB Magnetometer (Optional)}

\begin{lstlisting}[style=diagram]
Raspberry Pi 5          RM3100 / HMC5883L
--------------          -----------------
USB port     ---------- USB breakout board
\end{lstlisting}

Or via I2C if using a bare breakout:

\begin{lstlisting}[style=diagram]
RM3100 Board            Raspberry Pi 5 GPIO
------------            -------------------
VIN  ------------------- Pin 17 (3.3V)
GND  ------------------- Pin 9  (GND)
SDA  ------------------- Pin 3  (GPIO 2 -- I2C SDA)
SCL  ------------------- Pin 5  (GPIO 3 -- I2C SCL)
\end{lstlisting}

\subsection{Pi Camera (Optional)}

\begin{lstlisting}[style=diagram]
Pi Camera Module 3 (ribbon cable) -> Pi 5 CSI connector
\end{lstlisting}

\begin{enumerate}
  \item Lift the CSI connector latch on the Pi
  \item Insert the ribbon cable (contacts facing the board)
  \item Press the latch down
\end{enumerate}

Test:
\begin{lstlisting}[language=bash]
rpicam-still -o test.jpg
\end{lstlisting}


%% ============================================================
\section{Field Deployment}

\subsection{Site Selection}

\textbf{Ideal characteristics:}
\begin{itemize}
  \item Remote---minimal RF interference (away from cities, cell towers, WiFi)
  \item Open sky---unobstructed view, no overhead powerlines
  \item Soil ground---for spike coupling (not concrete/tarmac)
  \item Dark---for visual observation (no light pollution)
  \item Legal access---private land with permission, or public land where permitted
\end{itemize}

\textbf{UK suggestions:}
\begin{itemize}
  \item Brecon Beacons (dark sky reserve)
  \item Gower Peninsula (Darren---this is right on your doorstep)
  \item Snowdonia
  \item Scottish Highlands
  \item Any rural area away from major roads
\end{itemize}

\subsection{Setup Procedure}

\begin{lstlisting}[style=diagram]
Time: Allow 30 minutes for full setup

1. CHOOSE LOCATION
   - Flat ground, away from trees and structures
   - Good soil for ground spike (not rocky)

2. DEPLOY GROUND COUPLER
   - Drive ground spike into soil (hammer it in ~200mm)
   - Ensure plate sits flat and firm
   - Run speaker wire to amplifier location (2-5m)

3. SET UP ELECTRONICS
   - Place Pi, amp, and battery in case (or on dry surface)
   - Connect: DAC -> Amp -> Speaker wire -> Shaker
   - Connect: 12V battery -> Amp
   - Connect: 5V supply -> Pi

4. DEPLOY ANTENNAS (Tier 2)
   - TX antenna: mount on tripod, point upward
   - RX antenna: mount on separate tripod, 5m+ from TX
   - Connect SMA cables to HackRF and RTL-SDR
   - Ensure antennas are >2m from people during operation

5. POWER ON
   - Boot Pi (takes ~30 seconds)
   - SSH in: ssh pi@hlb.local

6. CAPTURE BASELINE (Tier 2)
   - Run: hlb --baseline
   - Wait 2-3 minutes for baseline capture

7. START PROTOCOL
   - Tier 1: hlb --duration 3600
   - Tier 2: hlb --rf --freq 433 --duration 3600
   - Full:   hlb --rf --freq hydrogen --duration 3600
\end{lstlisting}

\subsection{Field Layout}

\begin{lstlisting}[style=diagram]
                    N
                    ^
                    |
           +-------+-------+
           |   TX Antenna   |  <- On tripod, 2m high
           |   (Helical)    |    Pointed at sky
           +-------+-------+
                   | SMA cable (1m)
                   |
    +--------------+-------------------+
    |         EQUIPMENT                |
    |  +-----+ +------+ +-----+       |
    |  |Pi 5 | |Amp   | |Batt |       |
    |  |     | |12V   | |12V  |       |  <- In weatherproof case
    |  |HackRF |TPA3116| |LiFe |       |
    |  |RTL-SDR|      | |PO4  |       |
    |  +--+--+ +--+---+ +-----+       |
    |     |       |                    |
    +-----+-------+--------------------+
          |       |
          |       | Speaker wire (3-5m)
          |       |
          |  +----v------+
          |  |  Ground   |  <- Centre of deployment area
          |  |  Plate +  |    Spike driven into soil
          |  |  Shaker   |
          |  +-----------+
          |
          | SMA cable (5m)
          |
    +-----v-----+
    | RX Antenna |  <- On separate tripod, 5m+ from TX
    | (Discone)  |    2m+ high
    +-----------+


    OBSERVER: 10m+ from equipment
    +---------+
    |  You    |  <- Laptop with SSH, or just let it run
    |  (here) |    Bring chair, thermos, binoculars, camera
    +---------+
\end{lstlisting}


%% ============================================================
\section{Operating Procedures}

\subsection{Standard Session}

\begin{lstlisting}[language=bash]
# SSH into Pi
ssh pi@hlb.local

# Quick check
hlb --check

# Start (1 hour, mechanical only)
hlb --duration 3600

# Start (1 hour, full protocol, 433 MHz ISM legal)
hlb --rf --freq 433 --duration 3600

# Start (1 hour, full protocol, hydrogen line -- REQUIRES HAM LICENCE)
hlb --rf --freq hydrogen --duration 3600
\end{lstlisting}

\subsection{Automated Session (systemd)}

For unattended operation:

\begin{lstlisting}[language=bash]
# Create service file
sudo tee /etc/systemd/system/hlb.service << 'EOF'
[Unit]
Description=Hydrogen Line Beacon
After=network.target sound.target

[Service]
Type=simple
User=pi
WorkingDirectory=/home/pi
ExecStart=/home/pi/.local/bin/hlb --duration 7200 --programme full
Restart=on-failure
RestartSec=30
Environment=PYTHONUNBUFFERED=1

[Install]
WantedBy=multi-user.target
EOF

# Enable and start
sudo systemctl daemon-reload
sudo systemctl enable hlb.service
sudo systemctl start hlb.service

# Check status
sudo systemctl status hlb.service

# View logs
journalctl -u hlb.service -f
\end{lstlisting}

\subsection{Reading the Logs}

All events are logged to \texttt{./hlb\_logs/}:

\begin{lstlisting}[language=bash]
# Session summary
cat hlb_logs/session_*.json | python3 -m json.tool

# Event log (today)
cat hlb_logs/events_$(date +%Y%m%d).jsonl

# Count anomalies
grep '"type": "anomaly"' hlb_logs/events_*.jsonl | wc -l
\end{lstlisting}

\subsection{What to Record Manually}

Keep a field notebook (or phone notes) with:

\begin{itemize}
  \item Date, time, location (GPS coordinates)
  \item Weather conditions (clear, cloudy, temperature, wind)
  \item Moon phase
  \item Programme used and duration
  \item Any visual observations (lights, movement, anything unusual)
  \item Any sounds or sensations
  \item Any equipment anomalies (unexpected readings, interference)
  \item Timestamps of anything interesting
\end{itemize}


%% ============================================================
\section{Troubleshooting}

\subsection{No sound from shaker}

\begin{lstlisting}[language=bash]
# Check audio output
aplay -l
# Should list the I2S DAC

# Test with a tone
speaker-test -t sine -f 100 -l 1
# 100 Hz is audible -- if the shaker vibrates, wiring is correct

# Check I2S is enabled
cat /boot/firmware/config.txt | grep hifiberry
# Should show: dtoverlay=hifiberry-dac
\end{lstlisting}

\subsection{HackRF not detected}

\begin{lstlisting}[language=bash]
# Check USB
lsusb | grep -i hackrf
# Should show: "Great Scott Gadgets HackRF One"

# Check firmware
hackrf_info
# If error: try different USB cable or port

# Permissions
sudo usermod -aG plugdev pi
# Then log out and back in
\end{lstlisting}

\subsection{RTL-SDR not detected}

\begin{lstlisting}[language=bash]
# Check USB
lsusb | grep -i rtl
# Should show: "Realtek Semiconductor Corp. RTL2838"

# Check driver blacklist
cat /etc/modprobe.d/blacklist-rtl.conf
# Should show: blacklist dvb_usb_rtl28xxu

# If the DVB driver loaded first:
sudo modprobe -r dvb_usb_rtl28xxu
sudo modprobe rtl2832_sdr
\end{lstlisting}

\subsection{Low signal / weak vibration}

\begin{itemize}
  \item \textbf{Check amplifier gain}---turn the potentiometer on the TPA3116D2
  \item \textbf{Check 12V supply}---amp needs real 12V, not 5V from Pi
  \item \textbf{Check shaker mounting}---must be bolted tight to plate, not just resting
  \item \textbf{Check ground spike}---must be driven into real soil, not sitting on surface
  \item \textbf{Check speaker wire}---14 AWG minimum, no breaks, good connections
\end{itemize}

\subsection{Protocol shows no anomalies}

\begin{itemize}
  \item This is normal---anomalies are rare events
  \item Verify baseline was captured (check \texttt{hlb\_logs/baseline.json})
  \item Try lowering threshold: edit config to \texttt{anomaly\_threshold: 2.0}
  \item Ensure RX antenna is not too close to TX antenna ({>}5m separation)
  \item Check RTL-SDR is actually receiving (run \texttt{rtl\_power} manually)
\end{itemize}


%% ============================================================
\section{Safety}

\subsection{RF Exposure}

\begin{itemize}
  \item Keep all people \textbf{{>}2m from the TX antenna} during transmission
  \item At 100\,mW output, this distance provides a large safety margin
  \item The HackRF One's maximum output is 15\,dBm (\textasciitilde30\,mW) which is well within safe limits
  \item ISM band power limits are set specifically to be safe for incidental exposure
\end{itemize}

\subsection{Electrical}

\begin{itemize}
  \item The 12V amplifier circuit is low voltage and safe to touch
  \item Do not expose electronics to rain---use a waterproof case or cover
  \item Disconnect battery before modifying any wiring
  \item Use appropriate fuses on the 12V circuit (3A)
\end{itemize}

\subsection{Infrasound}

\begin{itemize}
  \item The mechanical channel produces sub-20\,Hz vibrations
  \item At the power levels used (10--50W shaker), these are well below harmful thresholds
  \item You may feel a mild tingling or vibration standing near the plate---this is normal
  \item Harmful infrasound levels ({>}120\,dB) require industrial equipment far beyond this build
\end{itemize}

\subsection{Lightning}

\begin{itemize}
  \item \textbf{Do not operate during thunderstorms}---the antennas are lightning attractors
  \item Take down antennas if storms are approaching
  \item Do not deploy on hilltops during unsettled weather
\end{itemize}

\subsection{UAP-Specific (per Skywatcher's Warnings)}

\begin{itemize}
  \item Skywatcher personnel report \textbf{historic injuries} from close UAP encounters including radiation exposure and directed energy
  \item If anything anomalous appears \textbf{at close range}, do not approach
  \item Consider bringing a \textbf{dosimeter} (radiation badge)---available for \textasciitilde\pounds20 on Amazon
  \item Maintain minimum \textbf{50m distance} from any observed anomaly
  \item Have a vehicle nearby for rapid departure if needed
  \item Tell someone where you are and when you'll be back
\end{itemize}

\subsection{Legal}

\begin{itemize}
  \item \textbf{433\,MHz / 868\,MHz:} Legal in UK/EU without licence at specified power levels
  \item \textbf{1.42\,GHz:} Requires amateur radio licence (Foundation minimum---\pounds50, one day)
  \item \textbf{Run \texttt{hlb --legal} for full details}
  \item You are responsible for compliance with Ofcom regulations
\end{itemize}

\vfill

\begin{center}
\rule{0.5\textwidth}{0.4pt}\\[6pt]
\textit{Mikoshi Ltd, 2026 --- MIT Licence}\\[4pt]
\textcopyright{} 2026 Mikoshi Ltd. MIT Licence.
\end{center}

\end{document}
